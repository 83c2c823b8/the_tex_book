\RequirePackage{luatex85}
\documentclass[leqno]{ltjsarticle}% leqnoは数式番号左
\usepackage{luatexja-fontspec}
%\setmainfont[Ligatures=TeX]{MS Mincho}
%\setmainjfont[YokoFeatures={JFM=prop},]{UD Digi Kyokasho NK-R}% デジタル教科書体
%\setmainfont{Gen Shin Gothic Regular}
\usepackage[top=10truemm,bottom=10truemm,left=20truemm,right=20truemm]{geometry}
\usepackage{luatexja} % ltjclasses, ltjsclasses のときは要らんらしい
\usepackage{multicol,amsmath,amssymb,mathtools,ascmac,amsthm,amscd,physics,comment,dcolumn,titlesec,mathrsfs,mypkg}%色々なパッケージ
\usepackage[all]{xy}% 何だったっけな
\usepackage{unicode-math}
\titleformat*{\section}{\Large\bfseries}
\setlength{\parindent}{0pt}% インデントくたばれ!
\pagestyle{empty}
\begin{document}
\textbf{\LARGE{The \TeX book}\ \Large{No.01}}\hspace{\fill} 
\section*{EXERCISE2.1}
  Alice said, ``I always use an en-dash insead of a hyphen when specifying page numbers like `480--491' in a bibliography.''
\section*{EXERCISE2.2}
  ----
\section*{EXERCISE2.3}

\end{document}